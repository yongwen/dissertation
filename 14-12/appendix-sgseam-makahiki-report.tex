
This document describes the improvement action report after applying the SGSEAM assessment to Makahiki. It is the deliverable for the final step of the SGSEAM process. It first describes the strengths and weaknesses of the Makahiki framework from the perspectives of different stakeholders, then suggests the actions to improve the Makahiki framework from these stakeholders' perspectives.

\autoref{table:assessment-result} lists the strengths and weaknesses of the Makahiki framework from the perspectives of  different stakeholders.

\begin{table}[ht!]
  \centering
  \begin{tabular}{|P{0.15\columnwidth}|P{0.78\columnwidth}|}
    \hline
    \tabhead{Stakeholder} &
    \tabhead{Assessment Result} \\
    \hline
    \begin{enumerate}[label={}, nosep, leftmargin=*]
    \item Player perspective 
    \end{enumerate}
    & 
    Strengths:
    \begin{enumerate}[nosep, leftmargin=*]
    \item literacy improvement is effective 
    \item found self-reported awareness and behavior effectiveness 
    \item player engagement level is high for the duration of 3-4 weeks
    \end{enumerate}               
     Weaknesses:
    \begin{enumerate}[nosep, leftmargin=*]
     \item reduction of energy consumption is small.
     \item player engagement is low for long duration.
     \item performance downgrade and some usability issues in the latest release.
     \end{enumerate}  \\
    \hline
    \begin{enumerate}[label={}, nosep, leftmargin=*]
    \item System admin perspective 
    \end{enumerate}
    & 
    Strengths:
    \begin{enumerate}[label={}, nosep, leftmargin=*]
    \item general good experience with improved cloud installation document 
    \end{enumerate}               
    Weaknesses:
    \begin{enumerate}[nosep, leftmargin=*]
    \item database installation documentation could be better
    \item issues in usability of the installation script
    \item difficulty in integrating an organization's LDAP and email server
    \item difficulty in using the SSL
    \end{enumerate} \\
    \hline
    \begin{enumerate}[label={}, nosep, leftmargin=*]
    \item Game designer perspective
    \end{enumerate}
    & 
    Strengths:
    \begin{enumerate}[label={}, nosep, leftmargin=*]
    \item most of the game design interface are easy to use 
    \end{enumerate} 
    Weaknesses:
    \begin{enumerate}[nosep, leftmargin=*]
    \item difficulty and lack of documentation in the predicate system 
    \item difficulty in generating event attendance code
    \item content creation is not WYSIWYG
    \item designing the smart grid game could be time consuming
    \end{enumerate} \\
    \hline
    \begin{enumerate}[label={}, nosep, leftmargin=*]
    \item Game manager perspective 
    \end{enumerate}
    & 
    Strengths:
    \begin{enumerate}[nosep, leftmargin=*]
    \item the submission approval interface is straight forward 
    \item the batch approval feature is useful 
    \item the game analytics ``Status'' page is very useful
    \end{enumerate} 
    Weaknesses:
    \begin{enumerate}[nosep, leftmargin=*]
    \item not easy to find the event confirmation code 
    \item the game site not available after the competition is over
    \item did not support automatically sending out game status emails 
    \end{enumerate}  \\
    \hline
    \begin{enumerate}[label={}, nosep, leftmargin=*]
    \item Game developer perspective
    \end{enumerate}
    & 
    Strengths:
    \begin{enumerate}[label={}, nosep, leftmargin=*]
    \item can be used to develop other serious games with less effort
    \end{enumerate} 
    Weaknesses:
    \begin{enumerate}[nosep, leftmargin=*]
    \item difficult to develop enhancements with the current documentation.
    \item documents have missing, confusing and inconsistent parts.
    \item steep learning curve. 
    \end{enumerate} \\
    \hline
  \end{tabular}
  \caption{Strengths and Weaknesses of Makahiki}
  \label{table:assessment-result}
\end{table}

\clearpage

The SGSEAM Improvement actions for Makahiki are shown in \autoref{fig:assessment-action}.

\begin{figure}[ht!]
\begin{mybox}

\textbf{Improvement Actions from Player stakeholder perspective:}
\begin{compactenum}
\item design a better way to measure the effectiveness of resource consumption 
\item research the benefits and approaches to maintain high level of engagements for long competition duration
\item investigate the performance downgrade in the latest release
\item investigate alternative to improve the rotating scoreboard display
\item fix the broken links to the educational video or provide a content validation tool to prevent content related issues
\end{compactenum}

\textbf{Improvement Actions from System Admin stakeholder perspective:}
\begin{compactenum}
\item address the database installation and configuration difficulty, either with better documentation or automate the process as much as possible
\item improve the usability of the installation script, with clear error message, redirect the verbose output into installation log
\item improve the documentation on LDAP and email server integration
\item automate the configuration of SSL as much as possible, improve documentation
\end{compactenum}

\textbf{Improvement Actions from Game Designer stakeholder perspective:}
\begin{compactenum}
\item improve usability in event confirmation code generation admin interface
\item provide a WYSIWYG content authoring and design tool
\item improve the documentation on how to use the predicates
\item fix the reported bugs 
\end{compactenum}

\textbf{Improvement Actions from Game Manager stakeholder perspective:}
\begin{compactenum}
\item improve usability of the interface for looking up event confirmation code
\item support the access to the game site after game over
\item support automatic emailing of game status periodically
\end{compactenum}

\textbf{Improvement Actions from Game Developer stakeholder perspective:}
\begin{compactenum}
\item create documentation widget development overview, data model, subclassing
\item improve documentation on existing structure, modules, APIs.
\item more examples or tutorials in developing with the framework
\end{compactenum}

\end{mybox}
\caption{SGSEAM Assessment Improvement Actions for Makahiki}
\label{fig:assessment-action}  
\end{figure}
