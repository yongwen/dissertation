\chapter{Conclusion}
\label{cha:conclusion}

This dissertation investigated the design, implementation, and evaluation of a serious game framework for sustainability called Makahiki and a stakeholder experience based assessment method for serious game framework. This chapter summarizes the results of the research, the contributions of the research, and possible future directions.

\section{Research Summary}
This research investigates the information technology infrastructure that can support effective and efficient development of serious games for sustainability. The research includes the development of an innovative serious game framework for sustainability that combining education and behavior change, and an assessment method accessing the effectiveness and efficiency of the IT infrastructure for serious games for sustainability with regarding the most important stakeholder's perspective.

%% TODO. more on SGSEAM

%% TODO. more on result of MAKAHIKI applications on several kukui cup

%% TODO. more on SGSEAM on makahiki result

\section{Contributions}

My research has generated several contributions, including: 1) Makahiki: open source information technology for development of serious games for sustainability; 2) SGSEAM: an assessment method for assessing serious game framework; 3) Evidence regarding the strengths and weaknesses of Makahiki as a serious game framework for sustainability; 4) Insights into creating and running several serious games in real-world;  5) Insights into hosting a serious game in the cloud infrastructure; and 6) Insights into the strengths and weaknesses of the assessment method.

\subsection{Makahiki: A serious game framework for sustainability}
TheMakahiki system described in Section 3.7 represents a contribution to the field of serious games
and energy competitions. Makahiki was developed over a period of two years by multiple developers.
My role in the Makahiki project was primarily articulating the requirements needed to support
the Kukui Cup, and as the primary internal tester. Along with George Lee, I worked on the user
evaluation of the Makahiki user interface through walkthroughs using mockups, in-lab user evaluations,
and external beta tests. Makahiki represents a unique type of serious game that combines
both data on energy consumption and game content related to energy literacy, breaking new ground
in this area.
Makahiki is an open source system hosted on GitHub [71], and consists of over 63,000 lines
of code. The design and evaluation of Makahiki is the subject of George Lee�s masters thesis [72],
and two conference papers for which I am coauthor [63, 10]. The version of Makahiki used in the
2011 UH Kukui Cup (Makahiki 1) is no longer under active development, having been replaced
by development of a new version (Makahiki 2) primarily by Yongwen Xu as part of his Ph.D.
dissertation, described later in Section 6.3.2.

\subsection{SGSEAM: An serious game framework assessment method}

\subsection{Strengths and weaknesses of Makahiki }

\subsection{Insights into creating and running several serious games in real-world}

\subsection{Insights into hosting a serious game in the cloud infrastructure}

\subsection{Insights into the strengths and weaknesses of the assessment method}

\section{Future Directions}

There are a variety of directions that can be pursued once this research is complete. This section discusses a variety of areas for future research.

\subsection{Applying SGSEAM to other serious game frameworks}
One of them is the evaluation of the SGSEAM itself. The design of SGSEAM creates a research question of what are the strengths and weaknesses of this assessment method. 
To better answer this question, SGSEAM should be applied to another serious game development environment. BuildingOS\cite{building-dashboard} by Lucid Design Group is such a serious game framework that is suitable for SGSEAM evaluation. Our research lab had made the effort to contact Lucid Design group for the collaboration. I created the assessment plan (\autoref{app:sgseam-lucid-guide}) for them which hope to minimize the effort spent from their side. But due to the workload of the Lucid design group, the collaboration did not continue. It is understandable that Lucid design group, as a newly found startup company, has other urgent endeavors to pursue. Looking for another suitable serious game framework to apply SGSEAM assessment is still an ongoing research direction.

\subsection{Community or Consortium to expand the content and game library}
Build a community to expand content and game library for Makahiki

Although the Kukui Cup now includes over 100 actions in its library of content, there are several
additional areas that could be expanded. The Kukui Cup currently lacks a video explaining the
important relationship between water and energy: many forms of energy generation require water,
and use of water requires energy to pump, heat, cool, and treat. We also lack videos delving into
Hawai�i�s options for future energy use, and a Native Hawaiian perspective on energy issues.
One issue with the Kukui Cup is that the educational content is largely of interest only until its
content has been assimilated. We do not anticipate that players would want to revisit most actions
unless they were able to earn additional points. This limited engagement is in contrast to games that
players enjoy playing over and over, such as the rich game environments Gee describes as being so
important for learning in games [49]. Some serious games such as the protein folding game Foldit
do manage to attract repeat players and meet their serious goals [64].
Beyond additional videos, the Kukui Cup could benefit from additional actions that are more
interactive in the way people traditionally think of games. Developing a complete game requires
much more effort than our current actions, but could potentially provide a much higher level of
engagement among players. One option would be to partner with developers of educational energy
148
games such as Energy City, a city simulation game where players pick must figure out how to supply
the energy needs of a growing city while minimizing environmental impact [48]. Figure 6.2 shows
a screenshot from Energy City.

\subsection{Scale and expand Makahiki}
 to support other geographical and cultural different locations.

All of the Kukui Cup challenges held to date have been held in institutions of higher education (the
participants in the East-West Center Kukui Cup were mostly students). While the student residence
hall environment has been a useful setting for our initial research, the Kukui Cup could have greater
impact if we expand to other settings.
One area we are actively exploring is deploying the Kukui Cup in the K-12 school environment.
Schools have certain similarities to the student residence halls, in that the participants would be
students, but many differences. In a K-12 environment, the Kukui Cup could be woven into existing
curricula, rather than being an extra-curricular activity. The Kukui Cup action content would
have to be redeveloped to be appropriate for the grade level being targeted, which is a substantial
undertaking. Potentially energy data could be measured on a per building or per classroom basis,
depending on the availability of smart metering infrastructure at the schools.

\subsection{Enhancement to Makahiki framework}
Lastly, Makahiki framework needs continuous enhancement. Some of the enhancement projects for Makahiki that we believe would be interesting and useful for the framework include:

\subsubsection{Real-time player awareness}
Currently it is not possible in Makahiki to know who is currently online and playing the game. Creating this awareness opens up new social gaming opportunities (performing tasks together), new opportunities for communication (chat windows), and potentially entirely new games (play against another online player). 

The goal of this enhancement is to extend the framework with a general purpose API that provides the identities of those who are online, and then the development of one or more user interface enhancements to exploit this capability.

\subsubsection{Deep Facebook integration}
Makahiki currently supports a ``shallow'' form of Facebook integration: you can request that your Facebook photo be used as your Makahiki profile picture, and you are given an opportunity to post to Facebook when the system notifies you of an accomplishment. 

One way to enhance the use of Facebook is to deepen the connection between user Facebook pages and their game play. This might involve more automated forms of notification (i.e. the same way ``Spotify'' playlists are posted to your Facebook wall), or ways in which your activities on Facebook could impact on your Makahiki challenge status. For example, posting a sustainability video to Facebook, or liking a Sustainability organization could earn you points. 

A different type of enhancement is to allow challenge designers to specify a Facebook page as the official Challenge Facebook information portal, and have the system automatically post information to that Facebook page as the challenge progresses.

\subsubsection{Action Library Management System}
Makahiki currently ships with over 100 possible  ``actions'' already developed for the Smart Grid Game. However, the current implementation suffers from a number of problems:

There is no convenient way to display and peruse the current set of actions. This has led to a duplicate representation of the smart grid game, implemented using a Google Docs spreadsheet linked to Google Sites pages. This approach has a lot of problems: it duplicates content, it does not provide a way to edit or manage content, it is already out of date.

The content is intimately tied to the Smart Grid Game implementation. The SGG is just one of many ways that the sustainability content could be presented to players. By separating ``content'' from the ``presentation'', more games can be developed using this content. We could enhance Makahiki to provide a ``content management system'' for ``actions'', which could involve the following changes to the current Makahiki system:
\begin{itemize}
\item Library actions are created, view, and edit via a separate admin interface which is defined for the content authoring role
\item An editor is provided to create action content and preview it in a formatted manner.
\item Library actions can be ``instantiated'' into the Smart Grid Game. 
\item Library content can be exported and imported into systems in order to support sharing. 
\item A public repository can be provided on GitHub. The format could be JSON.
\end{itemize}

