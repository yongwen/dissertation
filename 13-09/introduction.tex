\chapter{Introduction}
In this dissertation, I describe the Makahiki research project, which explores the information technology to provide infrastructure to facilitate the development of serious games for sustainability. The research consists of an open source serious game framework for sustainability and a method for assessing a serious game framework based on the stakeholder experience. In this chapter I explain the motivation of the research, briefly describe the Makahiki system and the SGSEAM assessment method,  finally summarize the results and contributions of this research.

\section{Motivation}
The rising cost, increasing scarcity, and environmental impact of fossil fuels
as an energy source makes a transition to cleaner, renewable energy sources an
international imperative. In Hawaii, the need for transition is especially acute, as the
state leads the US both in the price of energy and reliance on fossil fuels as an 
energy source (over 90\% from oil and coal \cite{hawaiienergypolicy}).

One barrier to this transition is the success that electrical utilities have had in
making energy ubiquitous, reliable, and easy to access, thus enabling
widespread ignorance in the general population about basic energy principles
and trade-offs. Moving away from petroleum is a technological, political, and social paradigm
shift, requiring citizens to think differently about energy policies, methods
of generation, and their own consumption than they have in the past.

On the other hand, changing people's behavior with respect to energy holds significant promise in reducing energy use. Darby's survey of energy consumption research found that identical homes could differ in energy use by a factor of two or more \cite{darby-review-2006}. Data from a military housing community on Oahu show energy usage for similar homes can differ by a factor of 4 \cite{Norton2010ZeroEnergyHomes}.

Petersen et al. describe their experiences deploying a real-time feedback
system in an Oberlin College dorm energy competition in 2005 that includes 22
dormitories over a 2-week period \cite{petersen-dorm-energy-reduction}. Web
pages were used to provide feedback to students. They found a 32\% reduction in
electricity use across all dormitories.

Over the past decade, running energy and water challenges have become a focal point for sustainability efforts at university and industry campuses, to facilitate and incentivize energy and water reduction. 
In a 2011 survey \cite{Hodge2010}, Hodge found that 163 universities and colleges held or planned to hold an energy competition during the 2010�2011 academic year. 40\% of these organizations are holding a competition for the first time. Hodge found that the average reduction in electricity use during these competitions is 9\%. 

Designers of such challenges typically have three choices for information technology: (a) build their own custom in-house solution (as was done at Oberlin College in 2006 \cite{petersen-dorm-energy-reduction}); (b) out-source to a commercial provider (as was done at the University of British Columbia in 2011 \cite{runkle2011dark}); or (c) use a minimal tech solution such as a web page and manual posting of data and results (as was done at Harvard University in 2012 \cite{harvard-green-cup}).

None of these choices are ideal: the custom in-house solution requires sophisticated design and implementation skills; out-sourcing can be financially expensive and impedes evolution; and the minimal tech solution does not fully leverage the possibilities of advanced information technology.

To provide a better alternative to these three choices, Makahiki is designed as an open source serious game framework for sustainability with the goal of providing the infrastructure to enable different organizations to easily create and manage sustainability related serious games including this kind of sustainability challenges. 

The closest technology to Makahiki of which we are aware is the Building Dashboard \cite{building-dashboard} that supports the Campus Conservation Nationals \cite{competetoreduce}, which has been used by over 100 schools nationwide in 2014 to implement an energy and water reduction competition, including the above mentioned University of British Columbia.  The Building Dashboard enables viewing, comparing and sharing building energy
and water use information on the web in compelling visual interface. But it focused on providing energy usage as passive information. There is little interaction between participants and the system. Unlike Makahiki, the Building Dashboard system does not support game mechanics, education, or synergy between real and virtual world environments.  In addition, it is not open source. The cost of the system creates the barrier for wider adoptions.

\section{Serious games and Gamification}

Another emergent issue is the explosive spread of game techniques, not only in
its traditional form of entertainment, but across the entire cultural spectrum.
Games have been shown with great potential as successful
interactive media that provide engaging interfaces in various serious contexts
\cite{mcgonigal2011reality,reeves2009total}. Priebatsch attempts to build a
game layer on top of the world with his location-based service startup
\cite{Priebatsch2010ted}. The adoption of game techniques to non-traditional areas such as finance,
sales, and education has become such a phenomenon that the Gartner Group
included ``gamification'' \cite{Deterding2011mt} on its 2011 Hype List.

A serious game, defined by Zyda, is ``a mental contest, played with a computer in accordance with specific rules that uses entertainment to further government or corporate training, education, health, etc''.  The system created by Makahiki consists of a number of smaller games which combined,  provides a game playing experience to the participants. So we consider that Makahiki framework produces a serious game, instead of a gamified educational application. While they are different, both gamification and serious games are trying to solve problems by influencing people's behavior utilizing both intrinsic and extrinsic motivation. Differentiating Makahiki from the other collegiate sustainability competitions are the game design elements in Makahiki. 

An Alternative Reality Game (ARG) is one type of serious game related to Makahiki in the way that they both  blend real and virtual world activities in the serious gaming context. McGonigal defines ARG as ``games you play to get more out of your real life, as opposed to games you play to escape it'' \cite{mcgonigal2011reality}. Her award winning serious ARG game ``World Without Oil'' \cite{worldwithoutoil} and ``Evoke'' \cite{urgentevoke} are designed with the goal of empowering people to come up with creative solutions to urgent real-world problems. Some players reported that the online game affected their real world behaviors \cite{mcgonigal2011reality}.

There are several sustainability related serious games.  Similar to games created by Makahiki, 
Vermontivate \cite{vermontivate} is a team-based game where  the players compete to accrue as many points as possible for their towns or schools by participating in a variety of sustainability-focused actions. Unlike Makahiki, Vermontivate does not have individual points or prizes and relies more on self-reported participation.

Reeves et al. described the design of Power House, an energy game that connects home smart meters to an online multiple player game with the goal to improve
home energy behavior \cite{Reeves2011powerhouse}. In the game, real world
energy data are transformed into a ``more palatable and relevant form of
feedback'', and players may be incentivized by the in-game rewards to complete
more energy-friendly real-world behaviors.

RecycleBank's ``Green Challenges'' \cite {recyclebank} is another serious game that used online gaming techniques to motivate participants to learn about green living and to take small green actions to live more sustainable lives offline.  According to the ``Gaming For Good'' report \cite {gamingforgood}, the challenge showed a 71\% increase in unique visitors and 97\% of participants surveyed said that the challenge increased
their knowledge about how to help the environment.

\section{Game Design and Motivation}

The research in serious games and gamification inspires the design of Makahiki in terms of benefits of a game, game design and how the game design can motivate and engage players in serious contexts. 

Why games? Results of a study published in the May 1998 issue of Nature \cite {koepp1998evidence} demonstrated that video game players experience regular releases of dopamine during game play. Dopamine is a neurotransmitter that signals pleasure rewards for food, sex and addictive drugs, such as cocaine. 

Psychology professor Mihaly Czikszentmihalyi introduced a specific kind of happiness that he named ``flow" \cite{csikszentmihalyi1991flow}, which is considered as one of the fundamental reasons that people play games \cite{murphygames}. Flow is a state of absorption, characterized by intense concentration, loss of self-awareness, a feeling of being perfectly challenged (neither bored nor overwhelmed).

Nicole Lazzaro argued that the use of extrinsic rewards will decrease the motivation to use your products and services once you remove that reward \cite {Lazzaro2011}. Vockell resonated that in education psychology, extrinsic motivators may lead to short-range activity increase but reduction in long-range interest in a topic. While intrinsic motivators motivate people best when they are working toward personally meaningful goals \cite{vockell2004educational}. 

In order to design a game that that is intrinsic motivated, Amy Jo Kim presented ``Smart Gamification'' which focuses on designing an effective ``Player Journey'' with intrinsic reward preferred over extrinsic reward \cite {Kim2010}. Kim pointed out that intrinsic values are greater than extrinsic rewards and ``a good game takes the player on a journey toward mastery".

The design of Makahiki is influenced by the above game design thinking. It combines extrinsic rewards such as prizes and achievement badges with intrinsic motivation such as improving our environments by learning. Makahiki employs Kim's player journey idea \cite{Kim2010} to design the on-boarding and progression of levels in the smart grid game. Other game mechanics, such as the ``appointment mechanics'' and the ``social interaction'' are also employed in the design of Makahiki.

\section{Serious game Framework assessment}

One of the benefits of using a game framework is that, if correctly designed, it will provide useful and
reusable ``building blocks'' with which to develop a variety of serious games. Yet how are we to know
 if a serious game framework has been ``correctly designed''?
 
In order to assess a serious game framework such as Makahiki, it is important to assess the serious games the framework produces.  One fundamental question in evaluating a serious game is the extent to which the
game achieves its ``serious'' purpose.  This is quite different from traditional entertainment games, in which evaluation focuses on usability or playability \cite{song2007new}. In the field of serious games, there is an increasing focus on the methodology of game evaluation \cite{Mayer2012233}. 

De Freitas and Oliver \cite{de2006can} point out that there are few frameworks to support the evaluation of education games. They introduce a four dimensional framework for evaluating 
educational games and simulations. The framework consists of: the context, the pedagogy, the representation, and the learner (or player).

These approaches focus on evaluation of a single serious game, as opposed to a serious game {\em \bf 
  framework}.  There exists some general purpose assessment tools such as GEQ (Game Engagement Questionnaire)\cite{brockmyer2009development} and QUIS (Questionnaire for User Interaction Satisfaction)\cite{harper1993improving} that are used for measuring game engagement and user interactions. 
  
During my literature search, I have not found any research in the area of assessment method for serious game framework. The design of SGSEAM is to provide an assessment method for the particular needs of assessing a serious game framework. 

\section{Research Description}

The overall research question this thesis investigates is:
What forms of information technology infrastructure can support effective and efficient development of serious games for sustainability?

In order to address this research question, I have designed two systems:
\begin{itemize}
    \item A software  infrastructure for development of serious games for sustainability.

    \item An assessment method that provides evidence regarding the strengths and weaknesses of the software infrastructure for the development of serious games for sustainability.
\end{itemize}

\subsection{Makahiki}

I developed an innovative serious game framework for sustainability called Makahiki, as a software infrastructure for the development of sustainability challenges. Makahiki explores one section of 
the design space where virtual world game mechanics are employed to affect real world sustainability
behaviors.  The ultimate goal of the Makahiki project is not just to affect behaviors during the course
 of the game, but also to produce long lasting, sustained change in behaviors and outlooks by participants.

 Makahiki has a unique feature set intended to foster more rapid innovation and development. These 
 features include: (1) an open source license and development model which makes the technology 
 available without charge and facilitates collaborative development and improvement; (2) support for an ``ecosystem'' of extensible, interrelated, customizable games and activities; (3) real-time game 
 analytics for research and evaluation; (4) pedagogically organized and extensible learning activities; 
 (5) a responsive user interface supporting mobile, tablet, and laptop displays; and (6) support for 
 deployment to the cloud as an inexpensive option for hosting the competition.

The Makahiki framework has been successfully used to create and manage seven real-world sustainability games by four organizations, in different environments and with different requirements. The instance are: 1) the Kukui Cup energy challenges at the University of Hawaii at Manoa in 2011, 2012 and 2014; 2) the HPU Kukui Cup energy challenges at the Hawaii Pacific University in 2012 and 2013; 3) the East West Center energy and water challenge in 2012; and 4) the Kukui Cup educational challenge at the Holy Nativity School in 2013.

\subsection{SGSEAM}

While these real world Makahiki instance provides some evidence regarding the ability of the Makahiki framework to support sustainability games in different environments, a more rigorous evaluation of Makahiki would yield better quality insight.  In order to assess the effectiveness and efficiency of software infrastructure for serious games for 
sustainability, I designed an assessment method called Serious Game Stakeholder Experience Assessment Method (SGSEAM). In a nutshell, SGSEAM (pronounced ``sig-seam'') identifies the most important stakeholders of a serious game framework and provides a method for gaining insight into the strengths
and shortcomings of the framework with respect to each stakeholders' needs. We consider
SGSEAM as an assessment method instead of an evaluation method. The main purpose of an
evaluation is to ``determine the quality of a program by formulating a judgement''
\cite{hurteau2009legitimate}. An assessment, on the other hand, is nonjudgmental. SGSEAM does
not try to judge a framework according to a standard, instead, it is used to identify the major
strengths and shortcomings of a framework so that the community could benefit from the
assessment by learning from the strengths and improving the shortcomings.

SGSEAM identifies the major stakeholders whose experiences affect a serious game framework as a software infrastructure as: players, system admins, game designers, game managers and game developers. For each stakeholder, multiple {\em in-vivo} and {\em in-vitro} assessment approaches can be used, depending on the resources available. The more approaches applied, the higher one's confidence in the accuracy of the assessment results.


\subsection{Evaluation}

Makahiki, as a serious game framework for sustainability, had been used by 4 different organizations to create a total of 7 serious game instances targeting to educate and foster sustainable behavior among the communities. The first Kukui Cup Energy challenge at the University of Hawaii at Manoa (UHM) was held in 2011 for 3 weeks for over 1,000 first year students living in the residence halls. UHM subsequently held the second and third Kukui Cup Energy challenges in 2012 and 2014 for different first year students and different durations, for 9 months and 2 weeks respectively. Hawaii Pacific University (HPU) held their 3-week Kukui Cup Energy challenge in Fall 2012 and 2013 for about 200 students each year. An international organization called the East-West Center (EWC) held a 2-week Kukui Cup Energy and Water challenge for the international residents living in the residence halls. An Hawaii private school called Holy Nativity School (HNS) held a pilot 7-month Kukui Cup challenge for the elementary school students. 

The real world case study of Makahiki is to look at the different requirements of these organizations, and how the corresponding different configurations in the Makahiki framework can be done to support these requirements, thus gain insight into how well the Makahiki framework can be used and customized in the real world scenarios. 

The successful creation of serious game challenges by four different organizations
provides evidence that Makahiki can be successfully tailored
to the needs of different organizations. First, UH and HPU used different metering
infrastructure, and EWC collected their resource data manually. HNS did not have energy data at all. 
Second, while UH and HPU
challenges involved only energy consumption data, the EWC challenge involved both energy
and water consumption data. Third, the IT infrastructure at UH and HPU provided
authentication services using CAS (Central Authentication Service) and LDAP, while EWC
used UH's CAS and the built-in Django authentication.  Fourth, the user interface was customized to
``brand'' each challenge with the logo, thematic elements, and the education contents of
the sponsoring organizations. Lastly, HPU instances were hosted in its local infrastructure, while UHM, EWC and HNS were hosted in the Heroku PaaS cloud. More details of the real world instance evaluation results are described in \autoref{sec:realworld-result} of the Result chapter.

In addition to the real world instance evaluation, I carried out a formal assessment of the Makahiki framework by apply the SGSEAM to Makahiki to determine the strengths and weaknesses of the framework from the five serious game stakeholders' experience perspectives. 

Following the SGSEAM process, the assessment plan (\autoref{app:makahiki-assessment-plan}) was created as the first deliverable of the process, which  guided the SGSEAM assessment of Makahiki. The assessment plan identifies the participants in the categories of five SGSEAM stakeholders. Two categories of assessment approaches were determined. The {\em in-vivo} assessment approaches of pre-post effectiveness study, in-game surveys and post-hoc interviews were applied to the real-world Kukui Cup challenges created by the Makahiki framework, to assess the experiences of players, system admins, game designers, and game managers. The {\em in-vitro} assessment approaches of in-lab study used the UHM ICS691 serious game course in Spring 2013 to assess the experiences in game installation, game design and game development. 

After creating the assessment plan, I carried out the Makahiki assessment according to the SGSEAM plan, a data repository (\autoref{app:makahiki-data-repository}) was created to collect all the assessment data. The data was analyzed to determine the strengths and weaknesses of the Makahiki framework. The results of the analysis are described in the \autoref{sec:assessment-result} of the Results chapter. The final deliverable of the SGSEAM assessment process was created as the improvement action report described in \autoref{app:makahiki-improvement-report}. 

The SGSEAM assessment report for Makahiki (\autoref{app:makahiki-improvement-report}) lists the strengths and weaknesses of Makahiki, and the recommended improvement actions from the five stakeholders' perspectives. SGSEAM revealed that the major strengths of Makahiki are the ability to create an engaging and effective serious game for an organization, and the ability to provide easy-to-use interface for game designers and managers. SGSEAM revealed the major weaknesses of Makahiki are difficulty in developing new enhancements to extend the framework from a game developer's perspective, difficulty to integrate external services such as LDAP and email server from a system admin's perspective, and the lack of WYSIWYG content authoring tool from a game designer's perspective.

The application of SGSEAM to Makahiki had effectively identified several strengths and weaknesses of the Makahiki as the serious game framework for sustainability. 


\section{Contribution}
The contributions of this research are:

\subsection{Makahiki: A serious game framework for sustainability}
	
The Makahiki framework creates a unique type of serious games that combine
both data on resource consumption and game content related to sustainability, represents an innovation in the area of serious game and sustainability education. Makahiki enables different organizations to easily create customized sustainability education and behavior change serious games that can be deployed in both local and cloud infrastructures. Makahiki is an open source system hosted on GitHub.

\subsection{SGSEAM: a serious game framework assessment method}
	
The SGSEAM contributes to the emerging field of serious game framework evaluation research. It describes a method for assessing serious game frameworks from the stakeholder experience perspective. The goal of SGSEAM is to identify the major strengths and weaknesses of a serious game framework, and provide the improvement actions to the framework developers. The benefits of SGSEAM assessment are that the developers of serious game frameworks can learn and improve from the findings of the assessment.
	
\subsection{Insights into the strengths and weaknesses of Makahiki}

The application of SGSEAM to Makahiki identified several strengths and weaknesses of  Makahiki as the serious game framework for sustainability. The strength and weakness report and improvement action report deliverable of this research provides valuable inputs to Makahiki developers for improving the framework, thus benefiting the community and organizations that use Makahiki as a tool for sustainability education and behavior change. 

\subsection{Insights into creating and running a variety of real-world serious games with Makahiki}

Makahiki contributed to the creation and running of seven real-world sustainability games by four  organizations. These game instances engaged the over 3,770 students from 4 educational institutes. These games were created in various configurations and hosting environments. The major differences involve the duration of the challenge, the population that could participate in the challenge, the type of resource(s) such as energy or water, whether they have smart meters installed, and the type of server hosting. Additional differences in requirements include the type of the authentication for participation and differences in game mechanics.  The successful creation of these serious game instances provides evidence that Makahiki can be successfully tailored to the needs of different organizations.

\subsection{Insights into managing cloud based serious games}
	
Hosting and managing the 4 Makahiki instances from 3 organizations in the cloud provided interesting experiences with cloud-based serious games. 
The successful use of Makahiki in the Heroku cloud platform provided a good alternative for organizations such as EWC and HNS who have less technology capacity. It also provides greater flexibility in infrastructure scaling up and down depending on the usage demand. In the context of serious games, which are often running for a period of time and have various usage demand throughout the game period, using the cloud infrastructure is a good alternative.

\subsection{Insights into the strengths and weaknesses of SGSEAM}

The application of SGSEAM to Makahiki effectively identified the strengths and weaknesses of Makahiki and provided improvement suggestions to Makahiki developers. It also revealed the strengths and weaknesses of SGSEAM as an assessment method.  The major strengths of SGSEAM includes the identification of the five categories of serious game stakeholders, a clearly defined three-step process and corresponding deliverables, and the flexibility of choosing from a set of additive assessment approaches. The major weaknesses of SGSEAM includes the process being expensive and time consuming and the sample size of the post-hoc interview approaches. Despite the limitations, the application of SGSEAM can still result in useful insights into the strengths and weaknesses of a serious game framework.

\section{Outline}

The dissertation is organized into the following chapters:

\begin{itemize}
	\item \autoref{cha:related-work} looks at related research, including serious game, gamification, serious game framework, and framework assessment.
	\item \autoref{cha:makahiki-design} describes the design and implementation of the Makahiki system.
    \item \autoref{cha:sgseam-design} describes the serious game framework assessment method SGSEAM.
	\item \autoref{cha:evaluation} describes evaluation of both Makahiki and SGSEAM.
	\item \autoref{cha:results} describes results of the evaluations.
	\item \autoref{cha:conclusion} concludes the dissertation with a list of contributions and future directions.
	\item \autoref{app:publication-list} contains the publications that have come out of this research that I have authored or co-authored.
	\item \autoref{app:makahiki-assessment-plan} describes the SGSEAM assessment plan for Makahiki.
	\item \autoref{app:makahiki-data-repository} describes the SGSEAM data repository for the Makahiki assessment.
	\item \autoref{app:makahiki-improvement-report} describes the SGSEAM improvement action report as the result of the Makahiki assessment.
	\item \autoref{app:sgseam-lucid-guide} describes the SGSEAM assessment guide written specifically for Lucid's serious game framework.
	\item \autoref{app:in-game-questionnaire} contains the survey questionnaire that was administered in the game to assess the player's experiences.
    	\item \autoref{app:googleform} describes the google forms that were used in the in-lab experiment studies.
\end{itemize}
